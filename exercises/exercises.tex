%\input{|"curl --stderr /dev/null https://github.com/danimalabares/config/preamble.sty"} 
%\input{|"curl --stderr /dev/null https://github.com/danimalabares/config/thms.sty"} 
%\input{|"curl --stderr /dev/null https://github.com/danimalabares/config/ops.sty"} 
\include{/Users/daniel/github/config/preamble.sty}
\include{/Users/daniel/github/config/thms.sty}
\include{/Users/daniel/github/config/ops.sty}

\begin{document}
{\LARGE algebraic topology exercises}
\tableofcontents
\section{Homework 1}
\setcounter{subsection}{-1}
\subsection{Preliminaries}
	In the category of sets there is a bijection $\Hom(X\times Y, Z)\cong \Hom(X, \Hom(Y, Z))$ that depends naturally on $X$, $Y$ and $Z$. The notions related to this bijection are “Cartesian closed category”, “currying” and “internal Hom”.
\begin{defn}
	A category $\Cc$ is \textbf{\textit{Cartesian closed}} if:
	\begin{enumerate}
		\item $\Cc$ has all finite products (Caveat: some require that $\Cc$ has all finite limits)
		\item For any object $Y$ the functor $- \times Y$ has a right adjoint, which we will denote by $\Map(Y,-)$ or by $-^Y$ .
	\end{enumerate}
\end{defn}
\begin{remark}
	By section 3 \href{https://ncatlab.org/nlab/show/internal+hom }{here}, the second property above implies that we get a functor $\Map(-,-) : \Cc^{\op} \times \Cc \to \Cc$, and moreover we get natural isomorphisms $\Hom(X, \Map(Y, Z)) \cong \Hom(X \times Y, Z)$ and $\Map(X, \Map(Y, Z))\cong \Map(X \times Y, Z)$.
\end{remark}

\begin{lemma}[Yoneda, \href{https://en.wikipedia.org/wiki/Yoneda_lemma}{wiki}]
	Let $F$ be a functor from a locally small category $\Cc$ to $\Set$. Then for each object $X$ of $\Cc$, the natural transformations $\Nat(\Hom(X,-),F)$ are in one-to-one correspondence with the elements of $F(X)$, that is
	\[\Nat(\Hom(X,-), F)\cong F(X)\]
	Moreover, this isomorphism is natural in $A$ and $F$ when both sides are regarded as functors from $\Cc\times\Set^{\Cc}$ to $\Set$. ($\Set^{\Cc}$ denotes de category of functors from $\Cc$ to $\Set$.)
	
	There is a contravariant version of Yoneda lemma asserting that if $F$ is a contravariant functor from $\Cc$ to $\Set$,
	\[\Nat(\Hom(-,X), F)\cong F(X).\]
\end{lemma}
\begin{coro}
	$\Nat (\Hom(-,X),\Hom(-,Y))\cong\Hom(X,Y)$.
\end{coro}
\begin{remark}
	The correspondence $X\mapsto\Hom(-,X)$ is fully faithful, that is, the correspondence $\Hom(X,X')\to\Nat(\Hom(-,X),\Hom(-,X'))$ is injective and bijective.
\end{remark}
\begin{exercise}[a]
	Let $\Cc$ be any category. Show that if for some objects $X$ and $X'$ we have $\Hom(X, Y )\cong \Hom(X', Y )$ for all objects $Y$, with isomorphisms being natural in $Y$, then $X\cong X'$. Dually, if $\Hom(Y, X) \cong \Hom(Y, X')$ naturally in $Y$ , then also $X\cong X'$.
\end{exercise}
\begin{proof}[Solution]
	The latter correspondence sends isomorphisms to isomorphisms. Since we are given a natural isomorphism in the problem, we conclude $X\cong X'$. The dual statement follows from the analogue formulation of Yoneda lemma.
\end{proof}
\begin{exercise}[b]
	Let $\Cc$ be a Cartesian closed category and $\pt$ be the terminal object. Show that for any object $X$ we have $X\cong\Map(\pt,X)$.
\end{exercise}
\begin{proof}[Solution]	
	Using item (a) with $X$ and $X'=\Map(\pt,X)$, it suffices to show that 
	\[\Hom(Y,X)\cong\Hom(Y,\Map(\pt,X))\]
	for all objects $Y$ and isomorphisms natural in $Y$.
	
	Since $\Cc$ is Cartesian closed, we have isomorphisms \href{https://ncatlab.org/nlab/show/internal+hom%20#properties}{natural} in $Y$ \[\Hom(Y,\Map(\pt,X))\cong\Hom(Y\times\pt,X)\cong\Hom(Y,X)\]
	since $\pt$ is a terminal object. Indeed:
	\begin{claim}
		In a Cartesian closed category $\Cc$ with terminal object $\pt$, we have that $Y\times\pt\cong Y$ for any object $Y$.
	\end{claim}
	\begin{proof}[Proof of claim] \textbf{(From \href{https://math.stackexchange.com/questions/542911/proving-basic-lemmas-about-categories-with-finite-products-and-terminal-initial}{StackExchange})} 
		The universal property of the product $Y\times\pt$ shows that the maps $\id_Y$ and $t_Y:Y\to\pt$ must factor through some $u:Y\to Y\times\pt$, making $\pi_1\circ u=\id_Y$.
		\[\begin{tikzcd}
			Y\arrow[rrd,bend left,"\id_Y"]\arrow[ddr,"t_Y",swap,bend right]\arrow[dr,dashed,"u"]\\
			&Y\times\pt\arrow[r,"\pi_1"]\arrow[d,"t_{Y\times\pt}"]&Y\\
			&\pt
		\end{tikzcd}\]
		 It is also true that $u\circ\pi_1=\id_{Y\times\pt}$, since \begin{itemize}
		 	\item $\pi_1\circ u\circ\pi_1=\id_Y\circ\pi_1=\pi_1$ and
		 	\item $t_{Y\times\pt}\circ u\circ\pi_1=t_{Y\times\pt}$
		 \end{itemize}
		 so by uniqueness of the universal property we get that $u\circ\pi_1=\id_{Y\times\pt}$.
		 \[\begin{tikzcd}
		 	Y\times\pt\arrow[dr,"\pi_1"]\arrow[rrrdd,bend left=45,"\pi_1"]\arrow[dddrr,bend right=60,swap,"t_{Y\times\pt}"]\\
		 	&Y\arrow[rrd,bend left,"\id_Y"]\arrow[ddr,"t_Y",swap,bend right]\arrow[dr,dashed,"u"]\\
		 	&&Y\times\pt\arrow[r,"\pi_1"]\arrow[d,"t_{Y\times\pt}"]&Y\\
		 	&&\pt
		 \end{tikzcd}\]
	\end{proof}
\end{proof}
\subsection{Based spaces and smash product}
\begin{defn}
	The appropriate analogue of the Cartesian product in the category of based spaces is the \textbf{\textit{smash product}} $X\wedge Y$ defined by
	\[X\wedge Y=X\times Y/X\vee Y.\]
	Here $X\vee Y$ is viewed as the subspace of $X\times Y$ consisting of those pairs $(x,y)$ such that either $x$ is the basepoint of $X$ or $y$ is the basepoint of $Y$.
\end{defn}
\begin{exercise}
	For a based space $(X,x_0)$ let $\Sigma X$ be $[0,1]\times X/\{1\}\times X\cup\{0\}\times X\cup[0,1]\times\{x_0\}$. Check that $\Sigma X\cong S^1\wedge X$. In particular $S^n\cong S^1\wedge S^{n-1}\cong (S^1)^{\wedge n}$.
\end{exercise}
\begin{remark}
	Another way of defining the reduced suspension $\Sigma X$ {\color{magenta}(I think)} is
	\[\Sigma X=(I\times X)/(t,x)\sim (0,y)\sim (1,y)\;\forall y\in X.\]
\end{remark}
\begin{proof}
	To see that $\Sigma X\cong S^1\wedge X$ simply notice that "both spaces are the quotient $X\times I$ with $X\times\partial I\cup\{x_0\}\times I$ collapsed to a point" (\cite{hatcher-at}, ex. 0.10). This is clear for $\Sigma X$. For $X\wedge S^1$, notice that collapsing $X\times\partial I$ to a point in $X\times I$ ammounts to taking $X\times S^1$ and collapsing one copy of $X$ to a point. Further, collapsing ${x_0}\times I$ to a point ammounts to collapsing the copy of $S^1$ in $X\vee S^1$ to a point.
	
	Let's try induction on $n$. If $n=2$, the smash product $S^1\wedge S^1$ is easily seen to be $S^2$ since it consists on collapsing the boundary $S^1\vee S^1$ of the square whose quotient yields $S^1\times S^1$. For the inductive step {\color{orange}Still incomplete…}
\end{proof}
\subsection{Mapping cylinders and Hurewicz cofibrations}
\begin{defn}[\href{https://en.wikipedia.org/wiki/Homotopy_extension_property}{wikipedia}]
	Let $X$ be a topological space and let $A\subset X$. We say that the pair $(X,A)$ has the \textbf{\textit{homotopy extension property}} if for any space $Y$, any homotopy $g_\bullet:A\to Y^I$ and any map $\tilde{g}_0:X\to Y$ such that $\tilde{g}_0\circ\iota=g_0$, there exists an \textbf{\textit{extension}} of $f_\bullet$ to a homotopy $\tilde{g}_\bullet:X\to Y^I$ such that $\tilde{g}_\bullet\circ\iota=g_\bullet$.
	\[\begin{tikzcd}
		A\arrow[r,"g_\bullet"]\arrow[d,swap,"\iota"]&Y^I\arrow[d,"\pi_0"]\\
		X\arrow[ur,dashed,"\tilde{g}_\bullet"]\arrow[r,swap,"\tilde{g}_0"]&Y
	\end{tikzcd}\]
	A \textbf{\textit{Hurewicz cofibration}} is a map $\iota:A\to X$ satisfying the homotopy extension property.
\end{defn}
\begin{exercise}[a] Prove that an inclusion $f:A\to X$ is a Hurewicz cofibration if and only if $A\times I\cup X\times \{0\}$ is a retract of $X\times I$.
\end{exercise}
\begin{remark}
	A little late I noticed the comment on Telegram that we may assume $A$ to be a closed subspace. Maybe I wouldn't have tried the solution following \cite{lectures} if I had knew this earlier, hehe--- still it was nice to see two different solutions.
\end{remark}
\begin{proof}[Solution following \cite{hatcher-at}]
	$(\implies)$ According to the former definition, choose $Y=(X\times\{0\})\cup(A\times I)$. The inclusion $A\times I\hookrightarrow Y$ is an homotopy $g_\bullet$ from $A$ to $Y$. Also, the inclusion $X\times\{0\}\hookrightarrow Y$ is an extension $\tilde{g}_0$. Then there exists an extension $\tilde{g}_\bullet$ of the whole homotopy, which is just a map from $X\times I$ to $Y$. We have thus produced a retraction:
	\[\begin{tikzcd}
		(X\times\{0\})\cup(A\times I)\arrow[d, hook]\arrow[r,"\id"]&(X\times\{0\})\cup(A\times I)=Y\\
		X\times I\arrow[ur,dashed]
	\end{tikzcd}\]
	
	$(\impliedby)$ Now suppose that $(X\times\{0\})\cup(A\times I)$ is a retract of $X\times I$. Let $Y$ be any space, $g_\bullet:A\to Y$ an homotopy and $\tilde{g}_0$ a map such that $\tilde{g}_0=g_0\circ f$.
	
	The homotopy $g_\bullet$ along with $\tilde{g}_0$ yield a map $\varphi:(A\times I)\cup (X\times\{0\})\to Y\cup (X\times\{0\})$. The key observation is that if $A$ is closed in $X$, then this map is continuous by the \href{https://en.wikipedia.org/wiki/Pasting_lemma#Formal_statement}{gluing lemma}. Then we simply compose the given retraction $r$ with this map to obtain the homotopy extension:
	\[\begin{tikzcd}
		A\times I\arrow[dd]\arrow[rr,"g_\bullet"]&&Y\\
		&(X\times\{0\})\cup(A\times I)\arrow[ur,"\varphi",swap]\\
		X\times I\arrow[ur,"r",swap]
	\end{tikzcd}\]
{	\color{red}A complicated argument in Hatcher's appendix shows that such a function is continuous even without the assumption that $A$ is closed.}
\end{proof}
\begin{proof}[Solution following \cite{lectures}]
	The homotopy extension property may be defined as a map $\iota:A\to X$ such that for any solid-arrow diagram as below, a dotted blue arrow exists making the whole diagram commute:
	\[\begin{tikzcd}
		A\arrow[r,"\iota"]\arrow[d]&X\arrow[d]\arrow[ddr,bend left]\\
		A\times I\arrow[r]\arrow[drr,bend right]&X\times I\arrow[dr,dashed,blue]&\\
		&&Y
	\end{tikzcd}\]
	Now consider the pushout corresponding to $\iota$ and the inclusion $A\to A\times I$. By the universal property of the pushout, the former diagram must factor by the pushout, and we get the following diagram:
	\[\begin{tikzcd}
		A\arrow[r,"\iota"]\arrow[d]&X\arrow[d]\arrow[ddr,bend left]\arrow[dddrr, bend left]\\
		A\times I\arrow[r]\arrow[ddrrr,bend right]\arrow[drr,bend right]&X\times\{0\}\sqcup_A(A\times I) \arrow[dr,dashed,"\exists!"]\arrow[ul,phantom,very near start,"\ulcorner"]&\\
		&&X\times I\arrow[dr,dashed,blue]\\
		&&&Y
	\end{tikzcd}\]
	The implication $(\implies)$ of our exercise again follows by choosing $Y=(X\times\{0\})\cup(A\times I)$. For the implication $(\impliedby)$ {\color{red}it appears that we have the same problem as before}: we need to construct the blue dashed arrow from the rest of the diagram (using that the black dashed arrow has a left inverse), but it seems that the natural thing to do is defining this function from the two pieces just like before, and we must make sure it is continuous.
\end{proof}
	\begin{defn}
		Let $f:X\to Y$ be a map. Let $M_f=X\times[0,1]\cup_fY$ be the \textbf{\textit{mapping cylinder of $f$}}, i.e. the pushout of $X\overset{\cong}{\to}X\times\{0\}\hookrightarrow X\times[0,1]$ and of $f:X\to Y$.
	Let $g:X\to M_f$ be the map $X\overset{\cong}{\to}X\times\{1\}\to M_f$. Let $h:M_f\to Y$ be the map that is induced by $X\times[0,1]\to Y:(x,t)\mapsto f(x)$ and $\id_Y:Y\to Y$. Observe that $f$ is the composition of $g$ and $h$.
	\end{defn}
	
\begin{remark}
		In both exercises below you might have to use the fact that pushouts are colimits and that colimits commute with products in $\CGWH$, i.e. $(\colim A_i)\times B$ is canonically homeomorphic with $\colim(A_i \times B)$.
\end{remark}
\begin{exercise}\leavevmode
	\begin{enumerate}
		\item[b.] Show that $h$ is a deformation retract, and in particular is a homotopy equivalence.\label{exer:1.2.a}
		\item[c.] Show that $g : X \to M_f$ is a cofibration. You may use exercise (a), but the direct proof might be simpler.
	\end{enumerate}
\end{exercise}

\begin{proof}[Solution]\leavevmode
	\begin{enumerate}
		\item[b.] We have that
		\[\begin{tikzcd}
			X\times\{0\}\arrow[r,"f"]\arrow[d,swap,"\id\times 1"]\arrow[rd,bend right,"g"]&Y\arrow[d]\arrow[ddr,bend left,"\id_Y"]\\
			X\times I\arrow[r,swap]\arrow[drr,bend right,swap,"(x\text{,}t)\mapsto f(x)"]&M_f\arrow[ul, phantom, "\ulcorner", very near start]\arrow[dr,dashed,"h"]\\
			&&Y
		\end{tikzcd}\]
		We must show that there is a homotopy between the identity map on $M_f$ and a retraction from $M_f$ to $Y$. So we want $h:M_f\times I\to M_f$ such that
		\[h(-,0)=\id_{M_f},\quad \img h(-,1)\subset Y\quad\text{and}\quad h(-,1)|_Y=\id_Y\]
		Since $M_f$ is a pullback, we can see it as a colimit, that is
		\[M_f=\colim(\begin{tikzcd}[column sep=small]
			X\times I&\arrow[l]X\arrow[r]&Y
		\end{tikzcd})\]
		and, since colimits commute with products in $\CGWH$, we get
		\[M_f\times I=\colim(\begin{tikzcd}[column sep=small]
			X\times I\times I&\arrow[l]X\times I\arrow[r]&Y\times I
		\end{tikzcd})\]
		that is,
		\[\begin{tikzcd}
			X\times\{0\}\times I\arrow[r]\arrow[d]&Y\times I\arrow[d,"?"]\arrow[ddr,bend left,"(y\text{,}s)\mapsto y"]\\
			X\times I\times I\arrow[r,"?",swap]\arrow[drr,bend right,"(x\text{,}t\text{,}s)\mapsto (x,t(s-1))",swap]&M_f\times I\arrow[ul, phantom, "\ulcorner", very near start]\arrow[dr,dashed]\\
			&&M_f
		\end{tikzcd}\]
		{\color{red}[I certainly got stuck in concluding…]}
		
		\item[c.] {\color{red}[Also in progress…]} Consider the following lifting problem:
		\[\begin{tikzcd}
			X\arrow[d,"g",swap]\arrow[r,"H"]&Z^I\arrow[d,"\pi_0"]\\
			M_f\arrow[ur,dashed]\arrow[r,"h",swap]&Z
		\end{tikzcd}\]
	\end{enumerate}
\end{proof}
\subsection{Path spaces and fibrations}
\begin{exercise}\leavevmode
	\begin{enumerate}[label*=\alph*.]
		\item Show that $\Map(I,Y)$ deformation retracts on $\Map(\pt,Y)$. Most likely you’ll have to find a correct map $I\times I\to I$. Also show that $\Map(I,Y)\to \Map(\pt,Y)$ is a Hurewicz fibration. The key map will be of the form $I \times I \to I \times I$.
	\end{enumerate}
\end{exercise}
\begin{proof}[Solution]\leavevmode
	\begin{enumerate}[label*=\alph*.]
		\item
		
		\textbf{($\Map(I,Y)\to\Map(\pt,Y)$ is a Hurewicz fibration.)} Let $A$ be any space. We must show that for any homotopy $H$ and lift $h_0$ there exists an homotopy $\tilde{H}$ as in the following diagram:
		\[\begin{tikzcd}
			A\times\{0\}\arrow[d,hook]\arrow[r,"h_0"]&\Map(I,Y)\arrow[d,"p"]\\
			A\times I\arrow[ur,dashed,"\tilde{H}"]\arrow[r,"H",swap]&\Map(0,Y)
		\end{tikzcd}\]
		From the isomorphism $\Map(X\times Y,Z)\cong\Map(X,\Map(Y,Z))$ we may rewrite the problem as
		\[\begin{tikzcd}
			(A\times\{0\})\times I\arrow[dr,"H_0"]\\
			(A\times I)\times I\arrow[r,dashed,"\tilde{H}"]&Y\\
			(A\times I)\times \{0\}\arrow[ur,"H",swap]
		\end{tikzcd}\]
		So we define the dashed arrow by
		\[(a,s,t)\mapsto\begin{cases}
			H_0(a,0,s-t)\qquad \text{when }s-t\geq0\\
			H(a,t-s,0)\qquad\text{when }s-t\leq0
		\end{cases}\]
		so that when $s=t$ the functions coincide, when $s=0$ we get $H$ and when $t=0$ we get $H_0$.
		
		\vspace{1cm}
		\textbf{($\Map(I,Y)$ deformation retracts on $\Map(\pt,Y)$.)} We must show there is a homotopy
		\[\begin{tikzcd}
			h:\Map(I,Y)\times I\arrow[r]&\Map(I,Y)
		\end{tikzcd}\]
		such that
		\begin{gather*}
			h(-,0)=\id_{\Map(I,Y)},\qquad h(-,1)\subset\Map(\pt,Y)\\\\ \text{and}\qquad h(-,1)|_{\Map(\pt,Y)}=\id_{\Map(\pt,Y)}.
		\end{gather*}
		Consider the map
		\begin{align*}
			I\times I&\to I\\
			(s,t)&\mapsto s-st
		\end{align*}
		
		Our deformation retract may be written like
		\iffalse\[\begin{tikzcd}[row sep=tiny]
			h:\Map(I,Y) \arrow[r]&\Map(I,\Map(I,Y))\\
			f(s)\arrow[r,maps to,shorten=2em]&t\mapsto f(s-st)
		\end{tikzcd}\]\fi
		\[\begin{tikzcd}[row sep=tiny]
			h:\Map(I,Y)\times I \arrow[r]&\Map(I,Y)\\
			(f(s),t)\arrow[r,maps to,shorten=1em]&f(s-st)
		\end{tikzcd}\]
		Then for $t=0$ we have the identity on $\Map(I,Y)$, and when $t=1$ we have $\ev_0$.
	\end{enumerate}
\end{proof}

Let $f : X \to Y$ be a map. Let $E_f$ be the pullback of $f : X \to Y$ and of $\ev_0 :\Map(I,Y)\to Y$. Let $h:X\to E_f$ be the map that sends $x$ to $(x,\operatorname{const}(f(x)))$, where $\operatorname{const}(f(x)) : I \to Y$ is the constant path at $f(x)$. Let $g : E_f \to Y$ be the composition of projection map $E_f \to \Map(I,Y)$ with $\ev_1 :\Map(I,Y)\to Y$.
\begin{exercise}
	\begin{enumerate}
		\item[b.] Show that $h:X\to E_{f}$ is an inclusion of a deformation retract.
		\item[c.] Show that $g:E_{f}\to Y$ is a fibration.
	\end{enumerate}
	
\end{exercise}


\section{Homework 1.5}
\subsection{Exercise on model categories}
\begin{exercise}[3.1.8 from \cite{riehl}]
	Verify that the class of morphisms $\Lc$ characterized by the left lifting property against a fixed class of morphisms $\Rc$ is closed under coproducts, closed under retracts, and contains the isomorphisms.
\end{exercise}
\begin{proof}[Solution]
	\textbf{(Coproducts.)}
	%{\color{magenta} Comment from Sergey: Coproduct of morphisms $A_i\to B_i$ in a category $\Cc$ is the obvious morphism $\sqcup A_i \to \sqcup B_i$. (Because in this construction morphisms $A_i\to B_i$ are seen as objects of what's called the arrow category of the category $\Cc$)}
	Suppose the maps $\ell_i:A_i\to B_i$ are in $\Lc$. Then their coproduct in the arrow category is the obvious map $\coprod A_i\to\coprod B_i$.
	
	Explicitly, their coproduct is an arrow $\coprod\ell_i$ and a collection of maps $f_i:\ell_i\to\coprod\ell_i$ such that for any other object $m:A\to B$ in the arrow category and a map $g:\ell\to m$, the following diagram is completed uniquely:
	\[\begin{tikzcd}[scale cd=1.2]
		\ell_i\arrow[r,"f_i"]\arrow[rr,bend right,"g",swap]&\coprod\ell_i\arrow[r,dashed,"\exists!"]&m
	\end{tikzcd}\qquad\forall i\]
	So we conclude that the source of $\coprod\ell_i$ is $\coprod A_i$ and its target $\coprod B_i$. Indeed, we really looking at
	\[\begin{tikzcd}[scale cd=1.2]
		A_i\arrow[r,"\ell_i"]\arrow[d,"f_i^1",swap]&B_i\arrow[d,"f_i^2"]\\
		\coprod A_i\arrow[r,"\coprod\ell_i"]\arrow[d,"\exists!",dashed,swap]&\coprod B_i\arrow[d,dashed,"\exists!"]\\
		A\arrow[r,swap,"m"]&B
	\end{tikzcd}\]
	
	Now consider the following lifting problem with respect to a morphism $r\in\Rc$:
	\[\begin{tikzcd}[row sep=large]
		\coprod A_i\arrow[d,swap,"\coprod\ell_i"]\arrow[r]&\bullet\arrow[d,"r\in\Rc"]\\
		\coprod B_i\arrow[r]&\bullet
	\end{tikzcd}\]
	Since $\ell_i\in\Lc$, we have maps
	\[\begin{tikzcd}[row sep=large]
		A_i\arrow[r]\arrow[d,"\Lc\ni\ell_i",swap]&\coprod A_i\arrow[d]\arrow[r]&\bullet\arrow[d,"r\in\Rc"]\\
		B_i\arrow[r]\arrow[rru,dashed]&\coprod B_i\arrow[r]&\bullet
	\end{tikzcd}\]
	which in turn means we have a unique map
	\[\begin{tikzcd}[row sep=large]
		A_i\arrow[r]\arrow[d,"\Lc\ni\ell_i",swap]&\coprod A_i\arrow[d]\arrow[r]&\bullet\arrow[d,"r\in\Rc"]\\
		B_i\arrow[r]\arrow[rru]&\coprod B_i\arrow[r]\arrow[ur,dashed]&\bullet
	\end{tikzcd}\]
	by the universal property of the coproduct $\coprod B_i$.
	
	To conclude we need to check that the triangles below and above the dashed arrow in the former diagram commute. This follows from the universal property of the coproducts $\coprod A_i$ and $\coprod B_i$ since, \href{https://en.wikipedia.org/wiki/Coproduct#Discussion}{in general},
	\[\Hom\left(\coprod X_i,Y\right)\cong\prod\Hom(X_i,Y).\]
	More explicitly, we now that the red paths in the following diagrams are the same:
	\[\begin{tikzcd}[row sep=large]
		A_i\arrow[r]\arrow[d,"\Lc\ni\ell_i",swap]&\coprod A_i\arrow[d]\arrow[r]&{\color{red}\bullet}\arrow[d,"r\in\Rc",red]\\
		{\color{red}B_i}\arrow[r,red]\arrow[rru]&{\color{red}\coprod B_i}\arrow[r]\arrow[ur,dashed,red]&{\color{red}\bullet}
	\end{tikzcd}\qquad
		\text{and}\qquad
	\begin{tikzcd}[row sep=large]
		A_i\arrow[r]\arrow[d,"\Lc\ni\ell_i",swap]&\coprod A_i\arrow[d]\arrow[r]&\bullet\arrow[d,"r\in\Rc"]\\
		{\color{red}B_i}\arrow[r,red]\arrow[rru]&{\color{red}\coprod B_i}\arrow[r,red]\arrow[ur,dashed]&{\color{red}\bullet}
	\end{tikzcd}\]
	and also
	\[\begin{tikzcd}[row sep=large]
		{\color{red}A_i}\arrow[r,red]\arrow[d,"\Lc\ni\ell_i",swap]&{\color{red}\coprod A_i}\arrow[d,red]\arrow[r]&{\color{red}\bullet}\arrow[d,"r\in\Rc"]\\
		B_i\arrow[r]\arrow[rru]&{\color{red}\coprod B_i}\arrow[r]\arrow[ur,dashed,red]&\bullet
	\end{tikzcd}\qquad
	\text{and}\qquad
	\begin{tikzcd}[row sep=large]
		{\color{red}A_i}\arrow[r,red]\arrow[d,"\Lc\ni\ell_i",swap]&{\color{red}\coprod A_i}\arrow[d]\arrow[r,red]&{\color{red}\bullet}\arrow[d,"r\in\Rc"]\\
		B_i\arrow[r]\arrow[rru]&\coprod B_i\arrow[r]\arrow[ur,dashed]&\bullet
	\end{tikzcd}\]
	so the conclusion follows from the former comment.
	
	\textbf{(Closed under retracts.)} Let us at least state what a retract of a morphism $g$ should be in the arrow category. Recall that a retract is just
	\[\begin{tikzcd}
		X\arrow[r]\arrow[rr,bend right,"\id_X",swap]&Y\arrow[r]&X
	\end{tikzcd}\]
	So in the arrow category we get
	\[\begin{tikzcd}
		\bullet\arrow[r]\arrow[rr,bend left]\arrow[d,"f"]&\bullet\arrow[d,"g"]\arrow[r]&\bullet\arrow[d,"f"]\\
		\bullet\arrow[r]\arrow[rr,bend right,swap,"\id"]&\bullet\arrow[r]&\bullet
	\end{tikzcd}\]
\end{proof}

\subsection{Hatcher's exercise on Whitehead's theorem}
\begin{thm}[Whitehead, \cite{may}]
	If $X$ is a CW complex and $e:Y\to Z$ is an $n$-equivalence, then $e_*:[X,Y]\to[X,Z]$ is a bijection if $\dim X<n$ and surjection if $\dim X=n$.
\end{thm}
\begin{thm}[Whitehead, \cite{may}]\label{thm:W2}
	An $n$-equivalence between CW complexes of dimension less than $n$ is a homotopy equivalence. A weak equivalence between CW complexes is a homotopy equivalence.
\end{thm}
\begin{thm}[Whitehead (4.5), \cite{hatcher-at}]
	If a map $f:X\to Y$ between connected CW complexes induces isomorphisms $f_*:\pi_n(X)\to\pi_n(Y)$ for all $n$, then $f$ is a homotopy equivalence. In case $f$ is the inclusion of a subcomplex $X\hookrightarrow Y$, the conlusion is stronger: $X$ is a deformation retract of $Y$.
\end{thm}
\begin{exercise}[Hatcher 4.1.12]
	Show that an $n$-connected, $n$-dimensional CW complex is contractible.
\end{exercise}
\begin{proof}[Solution]
	Just recall that $n$-connectedness means that $\pi_i(X)=0$ for all $i\leq n$, which means that $X$ is contractible by \cref{thm:W2}.
\end{proof}

\section{Homework 2}
\begin{defn}[H-space, \cite{hatcher-at} p. 281]
	$X$  is an \textit{\textbf{H-space}}, 'H' standing for Hopf, if there is a continuous multiplication map $\mu:X\times X\to X$ and an identitity element $e\in X$  such that the two maps $X\to X$ given by $x\mapsto\mu(x,e)$ and $x\mapsto\mu(e,x)$ are homotopic to the identity through maps $(X,e)\to(X,e)$. 
\end{defn}

\begin{exercise}[4.1.3]
	For an H–space $(X , x_0 )$ with multiplication $\mu : X \times X \to X$ , show that the group operation in $\pi_n(X,x_0)$ can also be defined by the rule $(f + g)(x) = \mu(f(x),g(x))$.
\end{exercise}
\begin{proof}[Solution]
	According to the \href{https://en.wikipedia.org/wiki/Eckmann–Hilton_argument}{Eckmann-Hilton argument}, we may show that $\pi_n(X,x_0)$ with the usual operation $+$ and the operation $\oplus$  given by $(f\oplus g)(x) = \mu(f(x),g(x))$ coincide if we manage to show that for all $a,b,c,d\in \pi_n(X,x_0)$
	\[
		(a+b)\oplus(c+d)=(a\oplus c)+(b\oplus d)
	.\]
This follows from definitions. Recall that for $f,g\in \pi_n(X,x_0)$,
\[
	(f+g)(s_1,s_2,...,s_n)=\begin{cases}
		f(2s_1,s_2,\ldots,s_n)\qquad &s_1\in [0,1/2]\\
		g(2s_1-1,s_2,\ldots,s_n)\qquad &s_1\in [1/2,1]
	\end{cases}
\]
so
\begin{align*}
	(a\oplus c)+(b\oplus d)&=\begin{cases}
		(a\oplus c)(2s_1,s_2,\ldots,s_n)\qquad &s_1\in [0,1/2]\\
		(b\oplus d)(2s_1-1,s_2,\ldots,s_n)\qquad &s_1\in [1/2,1]
	\end{cases}\\
		&=\begin{cases}
		\mu(a(2s_1,s_2,\ldots,s_n),c(2s_1,s_2,\ldots,s_n))\qquad &s_1\in [0,1/2]\\
		\mu(b(2s_1-1,s_2,\ldots,s_n),d(2s_1-1,s_2,\ldots,s_n))\qquad &s_1\in [1/2,1]
	\end{cases}\\
		&=\mu(a+b,c+d)\\
		&=(a+ b)\oplus(c+ d)
\end{align*}


\end{proof}

\begin{exercise}[4.1.19]
	Consider the equivalence relation $\simeq_w$ generated by weak homotopy equivalence: $X \simeq_w Y$ if there are spaces $X = X_1, X_2,\ldots, X_n = Y$ with weak homotopy equivalences $X_i\to X_{i+1}$ or $X_i\leftarrow X_{i+1}$ for each $i$. Show that $X\simeq_w Y$ iff $X$ and $Y$ have a common CW approximation.
\end{exercise}
\begin{proof}[Solution]
	$(\impliedby)$ Suppose $Z$ is a common CW approximation of $X$ and $Y$, that is, $Z$ is a CW complex and there are weak homotopy equivalences $Z\to X$ and $Z\to Y$. Then the sequence of spaces $X=X_1$, $Z=X_2$ and $Y=X_3$ shows that $X\simeq_wY$.
	
	$(\implies)$ Suppose $Z$ is a CW approximation of $X$ and let's show it can be made (somehow) into a CW approximation of $Y$. There is a weak homotopy equivalence $Z\to X$, and also a weak homotopy equivalence either $X=X_1\to X_2$ or $X=X_1\leftarrow X_2$. I wonder if this implies that the composition $Z\to X=X_1\to X_2$ is also a weak homotopy equivalence
\end{proof}

\begin{exercise}[4.2.1]
	Use homotopy groups to show that there is no retraction $\R P^n\to \R P^k$ for $n>k>0$.
\end{exercise}
\begin{proof}[Solution]
	Suppose there is a retraction
	\[\begin{tikzcd}
		\R P^k\arrow[r,hook]\arrow[rr,bend right,swap,"\id"]&\R P^n\arrow[r]&\R P^k
	\end{tikzcd}\]
	it induces isomorphisms
	\[\begin{tikzcd}
		\pi_i(\R P^k)\arrow[r]\arrow[rr,bend right,swap,"\cong"]&\pi_i(\R P^n)\arrow[r]&\pi_i(\R P^k)
	\end{tikzcd}\]
	and then just notice that that map is zero. Indeed, recall that the homotopy groups of a covering space are the same as the base (because the homotopy groups of the fibers are trivial), so we have that $\pi_k(\R P^k)\to\pi_k(\R P^n)=0\to\pi_k(\R P^k)$ will be zero. 
\end{proof}

\begin{exercise}[4.2.2]
	Show that the action of $\pi_1(\R P^n)$ on $\pi_n(\R P^n)\cong\Z$ is trivial for $n$ odd and nontrivial for $n$ even.
\end{exercise}
\begin{proof}[Solution]
	(I have used \href{https://math.stackexchange.com/questions/375968/the-action-of-the-group-of-deck-transformation-on-the-higher-homotopy-groups}{StackExchange} as suggested in Telegram chat, and also \href{https://pages.uoregon.edu/njp/hw7solutions.pdf}{this internet pdf}.)

	By exercise 4.1.4 the action of $\pi_{1}(\mathbb{R}P^{n})$ on $\pi_{n}(\mathbb{R}P^{n})$ can be identified with the action of deck transformations $G(S^{n})\cong \pi_{1}(\mathbb{R}P^{n})\cong \Z/2$. The nontrivial element in such a group is the antipodal map, and it acts on $\pi_{n}(\mathbb{R}P^{n})$ by concatenation with the map induced on homotopy by the deck transformation (and a change of base point isomorphism).

	Then we use Corollary 4.25, which says that the degree map $\pi_{n}(S^{n})\to\Z$ is an isomorphism. Then our action, which is a map $\pi_{1}(\mathbb{R}P^{n})\to \operatorname{Aut}\pi_{n}(S^{n})\overset{\operatorname{deg}}{\cong } \mathbb{Z}$ is trivial for $n$ odd and nontrivial for $n$ even since antipodal map multiplies degree by $(-1)^{n+1}$.
\end{proof}

\begin{exercise}[4.2.8]
	Show that the suspension of an acyclic CW complex is contractible.
\end{exercise}
\begin{proof}[Solution]
	(Warning: there are acyclic spaces with non-trivial homotopy groups.) Let's try to use Hurewicz theorem. Recall that by Freudenthal suspension theorem (coro 4.24) that if $X$ is $n$-connected, then $\pi_{k}(X)\to\pi_{k+1}(\Sigma X)$ is an isomorphism for $k\leq 2n$. This makes $\pi_1$ of the suspension trivial.
\end{proof}


\begin{exercise}[4.2.12]
	Show that a map $f:X\to Y$ of connected CW complexes is a homotopy equivalence if it induces an isomorphism on $\pi_1$ and if a lift $\widetilde{f}:\widetilde{X}\to \widetilde{Y}$ to the universal covers induces an isomorphism on homology. [The latter condition can be restated in terms of homology with local coefficientes as saying that $f_*:H_*(X;\Z[\pi_1X])\to H_*(Y;\Z[\pi_1Y])$ is an isomorphism]. 
\end{exercise}


\begin{exercise}[4.2.13]
	Show that a map between connected $n$-dimensional CW complexes is a homotopy equivalence if it induces an isomorphism on $\pi_i$ for $i\leq n$. [Pass to universal covers and use homology.]
\end{exercise}
\begin{proof}[Solution]
	Let $X$ and $Y$ be $n$-dimensional CW complexes and $f:X\to Y$ such that $f_*:\pi_i(X)\to\pi_i(Y)$ is an isomorphism for $i\leq n$. Let's try to use Hurewicz theorem, which states that a map between simply connected CW complexes is a homotopy equivalence if it induces isomorphisms on all homology groups.
	
	Consider the universal covers $\tilde{X}$ and $\tilde{Y}$, which are simply connected and also \href{https://math.stackexchange.com/questions/1148411/universal-covering-space-of-cw-complex-has-cw-complex-structure/1148918#1148918}{have CW structures}. By prop. 4.1, the cover projections induce isomorphisms in the homotopy groups for all $i\geq2$. By \href{https://math.stackexchange.com/questions/2206906/unique-map-of-universal-covering-space}{StackExchange} there is a unique lift $\tilde{f}$ to the universal covers making the diagram on the left commute, and by functoriality the diagram on the right also commutes.
	
	
%	If we can show that there is a map inducing isomorphisms $H_i(\tilde{X})\cong H_i(\tilde{Y})$ for all $i$, then we get an homotopy equivalence between the universal covers.
	
	\[\begin{tikzcd}
		\tilde{X}\arrow[d,swap,"p"]\arrow[r,"\tilde{f}"]&\tilde{Y}\arrow[d,"q"]\\
		X\arrow[r,"f",swap]&Y
	\end{tikzcd}\qquad\qquad
	\begin{tikzcd}
		\pi_i(\tilde{X})\arrow[r,"\tilde{f}_*"]\arrow[d,swap,"p_*"]\arrow[d,"\cong"]&\pi_i(\tilde{Y})\arrow[d,"q_*"]\arrow[d,swap,"\cong"]\\
		\pi_i(X)\arrow[r,"f_*",swap]\arrow[r,"\cong"]&\pi_i(Y)
	\end{tikzcd}\qquad i\geq2\]
	We conclude that $\tilde{f}$ {\color{red}is a weak homotopy equivalence}, and by prop. 4.21 it induces isomorphisms on homology groups. Finally, by Hurewicz theorem (coro. 4.33) it is an homotopy equivalence and so is $f$.
\end{proof}

\iffalse\begin{remark}
	About Riemann surfaces
	
		\[\begin{tikzcd}
		\widetilde{U\backslash x}\arrow[d,swap,"p"]\arrow[r,"\tilde{f}"]&\widetilde{W\backslash y}\arrow[d,"q"]\\
		U\backslash x\arrow[r,"f",swap]&W\backslash y
	\end{tikzcd}\qquad\qquad
	\begin{tikzcd}
		\pi_i(\tilde{X})\arrow[r,"\tilde{f}_*"]\arrow[d,swap,"p_*"]\arrow[d,"\cong"]&\pi_i(\tilde{Y})\arrow[d,"q_*"]\arrow[d,swap,"\cong"]\\
		\pi_i(X)\arrow[r,"f_*",swap]\arrow[r,"\cong"]&\pi_i(Y)
	\end{tikzcd}\qquad i\geq2\]
\end{remark}\fi

\begin{exercise}[4.2.15]
	Show that a closed simply connected 3-manifold is homotopy equivalent to $S^3$.
\end{exercise}
\begin{proof}[Solution]
	Since both $S^3$  and $M$ are simply connected, by Whitehead's theorem it suffices to construct a map $M\to S^3$ that induces isomorphisms on $\pi_n(X,x_0)$. To construct the map first notice that $M$ is 2-connected. To see that $\pi_2(M)=0$ we notice that $H^2(M)\cong H_1(M)\cong \pi_1^{\operatorname{ab}}(X)\cong 0$ by Poincar\'e duality. By Universal Coefficient Theorem {\color{magenta}(?)}, we see that {\color{magenta}(the free-torsion part is the same in homology and cohomology, yielding)} $H_2(M)=0$ too. Now we use Hurewicz theorem, which tells us that the first non-zero homotopy group is isomorphic to the first non-zero homology group via the Hurewicz map $h:\pi_3(M)\cong H_3(M)$. Further, since $M$  is simply-connected, it is orientable by prop. 3.25, and by thm 3.26 $H_3(M)\cong \Z$.

	The generator of $\pi_3(M)$ is the map we need to apply Whitehead's theorem. Indeed, it is a map $f:S^3\to M$ such that $h[f]=f_*(\alpha)$ with $\alpha$ a generator of $H_n(D^n,\partial D^n)$, is a generator of $H_3(M)$ by definition of the Hurewicz map. In other words, $f_*$ maps generator to generator and thus is an isomorphism. Since the other homotopy groups are zero, we are done.
\end{proof}

\begin{exercise}[4.2.31]
	(This was solved using comments from exercise lecture and again \href{https://pages.uoregon.edu/njp/hw7solutions.pdf}{this internet pdf}.)

	For a fiber bundle $F\to E\to B$ such that the inclusion $F\hookrightarrow E$ is homotopic to a constant map, show that the long exact sequence of homotopy groups breaks into split short exact sequences giving isomorphisms $\pi_n(B)\cong \pi_n(E)\oplus\pi_{n-1}(F)$. In particular, for the Hopf bundles $S^3\to S^7\to S^4$ and $S^7\to S^{15}\to S^8$  this yields isomorphisms
	\begin{align*}
		\pi_n(S^4)&\cong \pi_n(S^7)\oplus\pi_{n-1}(S^3)\\
		\pi_n(S^8)&\cong \pi_n(S^{15})\oplus\pi_{n-1}(S^7)
	\end{align*}
	Thus $\pi_{7}(S^4)$ and $\pi_{5}(S^{15})\oplus\pi_{n-1}(S^7)$  contain $\Z$  summands.	
\end{exercise}
\begin{proof}[Solution]
	Consider the long exact sequence in homotopy,
	\[
		\begin{tikzcd}[column sep=small]
				\cdots\arrow[r]&\pi_i(F)\arrow[r]&\pi_i(E)\arrow[r]&\pi_i(B)\arrow[r]&\pi_{i-1}(F)\arrow[r]&\pi_{i-1}(E)\arrow[r]&\cdots
		\end{tikzcd}
	.\]
This yields
\[
\begin{tikzcd}[column sep=small]
	0\arrow[r]&\pi_i(E)\arrow[r]&\pi_i(B)\arrow[r]&\pi_{i-1}(F)\arrow[r]&0
\end{tikzcd}
.\]
To show that this sequence splits we show there is an arrow that goes backward in the last part of the previous sequence. Let $f$ represent a homotopy class in $\pi_{i-1}(F)$ and construct the following diagram:
\[\begin{tikzcd}[column sep=small]
	S^{n-1}\arrow[r,"f"]\arrow[d]&F\arrow[d]\\
	D^n\arrow[d]\arrow[r,dashed]&E\arrow[d]\\
	D^n/S^{n-1}\cong S^n\arrow[r]&B
\end{tikzcd}\]
where the dashed arrow is defined by
\[D^{n} \ni v\mapsto \begin{cases}
	H(f(v/|v|),|v|)\qquad &v\neq 0\\
	\text{basepoint} \qquad &v=0
\end{cases}\]
with respect to a nullhomotopy $H:F\times I\to E$ of the inclusion $F\hookrightarrow E$.
\end{proof}


\begin{exercise}[4.2.32]
	Show that if \begin{tikzcd}[column sep=small]
		S^k\arrow[r]&S^m\arrow[r]&S^n
	\end{tikzcd} is a fiber bundle, then $k=n-1$ and $m=2n-1$. [Look at the long exact sequence of homotopy groups.]
\end{exercise}
\begin{proof}[Solution]
	From the previous exercise we have
	\[
	\pi_i(S^n)=\pi_i(S^m)\oplus\pi_{i-1}(S^k)
	.\]
	Notice that the inclusion of the fiber in the total space \textit{is} homotopic to a constant map because this is a fibration, ie. there are local neighbourhoods in the base where the preimage looks like $\R^n\times S^n$, implying that $n+k=m$, that is, $k<m$. So $\pi_k(S^m)=0$.

	Now if we take $i=n$, we get that
	\[
	\Z\cong \pi_{n}(S^m)\oplus\pi_{n-1}(S^k)
	.\]
	Now observe that
	\begin{itemize}
		\item If $k=0$ and $n=m>1$ then $S^n=S^m$ is simply-connceted and there is no non-trivial covering $S^m\to S^n$.
		\item $k=0$  and $m=n=1$, then there is $S^0\to S^1\to S^1$.
		\item $k>0$ then $n<m$, so $\pi_{n}(S^m)=0$ and then $\Z\cong \pi_{n-1}(S^k)$. This means that $n-1\geq k$.
	\end{itemize}

	Now choose $i=k+1$. We get that
\[
\pi_{k+1}(S^n)\cong \pi_{k+1}(S^m)\oplus\Z
.\]
This means that, since $m=n+k$ because fiber bundle, we have $n+k\geq 2k+1>k+1$. This implies that $\pi_{k+1}(S^m)=0$. Finally $\pi_{k+1}(S^n)\cong \Z\implies k+1\geq n$.
\end{proof}

\begin{exercise}[4.2.34]
	Let $p:S^3\to S^2$ be the Hopf bundle and let $q:T^3\to S^3$ be the quotient map collapsing the complement of a ball in the 3-dimensional torus $T^3=S^1\times S^1\times S^1$ to a point. Show that $pq:T^3\to S^2$ induces the trivial map on {\color{magenta}$\pi_*$} and {\color{magenta}$\widetilde{H}_*$}, but is not homotopic to a constant map.
\end{exercise}
\begin{proof}[Solution]
	First let's show that $pq$ induces a trivial map on $\pi_*$ and $\widetilde{H}_*$. Recall that the product behaves good in homotopy groups, so that $\pi_{1}(T^3)\cong \Z^3$ and $\pi_{i}(T^3)\cong 0$ for $i>1$.

	Now, notice that fiber bundles are Hurewicz fibrations (over second countable manifolds). This gives us a lift
	\[\begin{tikzcd}
		T^3\arrow[r]\arrow[d]&S^3\arrow[d]\\
		T^3\times I\arrow[ur,dashed]\arrow[r]&S^2
	\end{tikzcd}\]
	We get a map $g:T^3\to S^3$ that factors through the fiber
\[\begin{tikzcd}
	g:H_3(T^3)\arrow[rr]\arrow[dr]&&H_3(S^3)\arrow[dl]\\
	&H_3(S^1)
\end{tikzcd}\]
which makes $f_*:H_3(T^3)\to H_3(S^3)$ an isomorphism.

\end{proof}


\begin{exercise}
	There is a fiber sequence $\U(n)\hookrightarrow\U(n+1)\to\U(n+1)/\U(n)\cong S^{2n+1}$. Use this to show that $\pi_k(\U(n))\to\pi_k(\U(n+1))$ is isomorphism for $n>k/2$. Compute $\pi_k(\U(n))$ for $n\geq 2$ and $k=1,2,3$. In fact, if $k$ is even then $\pi_k(\U(N))=0$ and if $k$ is odd then $\pi_k(\U(N))=\Z$, where again $N>k/2$. These equalities are known as Bott periodicity.
\end{exercise}
\begin{proof}[Solution]
	The required isomorphisms $\pi_k(\U(n))\to\pi_k(\U(n+1))$ follow simply from the fact that $S^{2n+1}$ is $2n+1$-connected: in the long homotopy sequence of the fiber bundle we have
	\[\begin{tikzcd}
		\pi_{k+1}(S^{2n+1})\arrow[r]&\pi_k(\U(n))\arrow[r]&\pi_k(\U(n+1))\arrow[r]&\pi_k(S^{2n+1})
	\end{tikzcd}\]
	so when $2n+1>k+1\iff n>k/2$ the homotopy groups of the spheres vanish and we have an isomorphism.
	
	The group $\pi_1(\U(n))$ is isomorphic to $\Z$. This follows from the fact that $\U(1)$ is homeomorphic to a circle and by induction using the former isomorphism $\pi_1(\U(n))\cong\pi_1(\U(n+1))$. We also have $\pi_2(\U(1))=0$, so that again by induction we get $\pi_2(\U(n))=0$. Finally, a similar argument shows $\pi_3(\U(n))=0$. Notice $n$ must be at least 2 for the isomorphism to work, so that to compute the first and second homotopy groups of $\U(n)$ we must use the exact sequences
\[\begin{tikzcd}
	0\arrow[r]&\SU(2)\arrow[r]&\S(n)\arrow[r]&\S(1)\arrow[r]&0
\end{tikzcd}\]
which yield $\pi_{2}(\U(n))=\mathbb{Z}$ and $\pi_{1}(\U(n))=0$.
\end{proof}


\section{Cohomology ring of $\C P^n$}
	\begin{exercise}
	Show that
	\[H^\bullet(\C P^n)\cong \Z[\alpha]/(\alpha^{n+1})\]
	where $\alpha$ has degree 2.
\end{exercise}
\begin{proof}
	The CW structure of $\C P^n$ consists of one cell for every even dimension. This gives us the following chain complex:
	\[\begin{tikzcd}
		\Z\arrow[r]&0\arrow[r]&\Z\arrow[r]&\ldots\arrow[r]&\Z,\qquad\text{if }n\text{ is even}
	\end{tikzcd}\]
	\[\begin{tikzcd}
		\Z\arrow[r]&0\arrow[r]&\Z\arrow[r]&\ldots\arrow[r]&0,\qquad\text{if }n\text{ is odd}
	\end{tikzcd}\]
	which yields the cohomology
	\[H^i(\C P^n)=\begin{cases}
		\Z,\quad&i=0,2,4,\ldots 2n\\
		0,\quad&i\text{ otherwise}
	\end{cases}\]
	so that
	\begin{align*}
		H^\bullet(\C P^n)&=H^0(\C P^n)\oplus H^2(\C P^n)\oplus\ldots\oplus H^{2n}(\C P^{n})
	\end{align*}
	This means that the underlying group of the cohomology ring is the same as that of
	\[\Z[\alpha]/(\alpha^{n+1})\]
	where $\alpha$ has degree 2. To show that these groups are also isomorphic as algebras we can use Poincar\'e duality as follows.
	
	Consider the case $n=2$, where we may immediately multiply the generator of second cohomology group with itself:
	\begin{align*}
		H^2(\C P^2)\times H^2(\C P^2)&\to H^4(\C P^2)\\
		(\alpha,\alpha)\qquad\quad&\mapsto\alpha\smile\alpha =\alpha^2
	\end{align*}
	By Poincar\'e duality this map is a nondegenerate symmetric bilinear form, so it must map generator to a generator. {\color{blue-violet}The fact that the product of the generator in degree 2 is the generator of degree 4} yields an homomorphism
	\iffalse\begin{align*}
		\varphi:H^\bullet(\C P^n)&\to\Z[\alpha]/(\alpha^{n+1})\\
		H^2(\C P^n)\ni\alpha&\mapsto\alpha
	\end{align*}
	satisfying
	\[\varphi(\alpha\smile\alpha)=\varphi()\]\fi
	\begin{align*}
		\varphi:\Z[\alpha]&\to H^\bullet(\C P^n)\\
		\alpha&\mapsto\alpha\in H^2(\C P^n)
	\end{align*}
	with kernel $(\alpha^{n+1})$ as desired.
	
	Now the case of $\C P^3$ is:
	\begin{align*}
		H^2(\C P^3)\times H^4(\C P^3)&\to H^6(\C P^3)\\
		(\alpha,\alpha^2)\qquad\quad&\mapsto\alpha\smile\alpha^2=\alpha^3
	\end{align*}
	which also  maps generator to generator, producing the desired algebra isomorphism. Notice we have used the group isomorphism $H^4(\C P^3)\approx H^4(\C P^2)$ when denoting the generator of $H^4(\C P^3)$ as $\alpha^2$. Such an isomorphism is induced by inclusion $\C P^{n-1}\hookrightarrow\C P^n$ via relative cohomology exact sequence.
	
	The case for dimension $n$ follows by induction.
\end{proof}

\section{Spectral sequences}\label{sec:spectral sequences}
\subsection{Wang spectral sequence}\label{ssec:Wang spectral sequence}

Let's use course notes from S. Burkin and \href{https://en.wikipedia.org/wiki/Spectral_sequence#Wang_sequence}{wikipedia} to understand Wang spectral sequence.

We will use the following result:

\begin{thm}[Serre spectral sequence for homology]
	Let $F\to X\to B$ be a fibration with $B$ path-connected. If $\pi_1(B)$ acts trivially on $H_*(F;G)$, then there is a spectral sequence $\{E^r_{p,q},d_r\}$ with
	\begin{itemize}
		\item
			\[d^r_{p,q}:E^r_{p,q}\to E^r_{p-q,q+r-1}\qquad \text{and} \qquad E^{r+1}_{p,q}=\ker d^r_{p,q}/\img d^r_{{\color{magenta}?}}\]
		\item The groups
			\[F_pH_n:=\img(H_n(X_p)\to H_n(X))\]
		where the map $H_{n}(X_p)\to H_{n}(X)$ is just the induced map by inclusion $X_p\hookrightarrow X$, form a filtration
			%\item Stable terms $E^\infty_{p,n-p}$ are isomorphic to the successive quotients $F_n^p/F^{p-1}_n$ in a filtration
		\[0\subset F_0H_n\subset \ldots\subset F_nH_n=H_n(X;G)\]
		of $H_n(X;G)$ such that
		\[E^\infty_{p,n-p}\cong F_pH_n/F_{p-1}H_n.\]
		Another way to write this is
		\[H_{p,q}^\infty=F_pH_{p+q}\Big/F_{p-1}H_{p+q}.\]
	\item \[E^2_{p,q}\cong H_p(B;H_q(F;G))\]
\end{itemize}
\end{thm}


To start consider a fibration over a sphere $F\hookrightarrow E\to B$. According to our theorem for Serre spectral sequences, we know that the $n$-th page looks as follows:
\begin{align*}
\begin{array}{c|cccccccc}
	n&H_n(F)&&&&&&&H_{n}(F)\\
	 &\vdots&&&&&&&\vdots\\
	2&H_2(F)&&&&&&&H_2(F)\\
	1&H_1(F)&&&&&&&H_1(F)\\
	0&H_0(F)&&&&&&&H_0(F)\\
	\hline\\
	 &0&&&\ldots&&&&n
\end{array}
\end{align*}
We can see there can be nontrivial differentials only on the $n$th page, so that $E^{n+1}=E^n$. Moreover, the differentials that don't vanish is are of the form
\[\begin{tikzcd}
E^n_{n,q}\arrow[r,"d^n_{n,q}"]&E^n_{0,q+n-1}
\end{tikzcd}\]
For $d_{n,q-n}^n$ we obtain
\[\begin{tikzcd}
	0\arrow[r]&\ker d^n_{n,q-n}\arrow[r,"\text{inclusion} ",hook]&E^n_{n,q-n}\arrow[r,"d^n_{n,q-n}"]&E^n_{0,q-1}\arrow[r,"\text{quotient}",two heads]&\operatorname{coker}d^n_{n,q-n}\arrow[r]&0
\end{tikzcd}\]
But all that is just
\[\begin{tikzcd}[row sep=small]
	0\arrow[r]&\ker d^n_{n,q-n}\arrow[r,"\text{inclusion} ",hook]\arrow[d,equals]&E^n_{n,q-n}\arrow[r,"d^n_{n,q-n}"]\arrow[d,equals]&E^n_{0,q-1}\arrow[r,"\text{quotient}",two heads]\arrow[d,equals]&\operatorname{coker}d^n_{q-n}\arrow[r]\arrow[d,equals]&0\\
	0\arrow[r]&\dfrac{\ker d^n_{n,q-n}}{\cancelto{0}{\img d^{n-1}_{n,q-n}}}\arrow[r]\arrow[d,equals]&E^2_{n,q-n}\arrow[r]\arrow[d,equals]&E^2_{0,q-1}\arrow[r]\arrow[d,equals]&\dfrac{E^n_{n,q-n}}{\img d^n_{n,q-n}} \arrow[d,equals]\arrow[r]&0\\
	0\arrow[r]&E^\infty_{n,q-n}\arrow[r]&H_{q-n}(F)\arrow[r]&H_{q-1}(F)\arrow[r]&E^{\infty}_{0,q-1}\arrow[r]&0
\end{tikzcd}\]
This is the first "half" of the Wang sequence. For the other half recall that by the theorem of Serre spectral sequence for homology we know that
\[E^{\infty}_{p,n-p}=F_pH_n/F_{p-1}H_n\]
for a filtration on the $n$-th homology group of the total space 
\[\varnothing=F_{-1}\subset F_0 \subset F_1\subset \ldots\subset F_n=H_{n}(E).\]
And recall that we may also write
\[E^\infty_{p,q}=F_pH_{p+q}/F_{p-1}H_{p+q}.\]
Then we can see that
\iffalse\[\begin{tikzcd}[row sep=small]
	0\arrow[r]&E^\infty_{0,p+n-1}\arrow[d,equal]\arrow[r]&H_{p+n-1}(E)\arrow[r]&E^\infty_{n,q-1}\arrow[r]\arrow[d,equal]&0\\
		  &\dfrac{F_0H_{p+n-1}(F)}{\cancelto{0}{F_{-1}H_{p+n-1}(F)}} &&\dfrac{F_nH_{p+n-1}}{F_{n-1}H_{p+n-1}}
\end{tikzcd}\]\fi
\[\begin{tikzcd}[row sep=small]
	0\arrow[r]&E^\infty_{0,q}\arrow[d,equal]\arrow[r,"i^*"]&H_{q}(E)\arrow[r]&E^\infty_{n,q-n}\arrow[r]\arrow[d,equal]&0\\
		  &\dfrac{F_0H_{q}(F)}{\cancelto{0}{F_{-1}H_{q}(F)}} &&\dfrac{F_nH_{q}(E)}{F_{n-1}H_{q}(E)}
\end{tikzcd}\]
But 
\[E^\infty_{0,q}=\dfrac{F_0H_{q}(F)}{\cancelto{0}{F_{-1}H_{q}(F)}}{\color{magenta}\quad \overset{\text{why?}}{=}}H_{q}(F)\]
\iffalse
	these groups are just {\color{magenta}(why?)}
\[\begin{tikzcd}[row sep=small]
	0\arrow[r]&E^\infty_{0,q}\arrow[d,equal,magenta]\arrow[r]&H_{q}(E)\arrow[r]\arrow[d,equal]&E^\infty_{n,q-n}\arrow[r]\arrow[d,equal,magenta]&0\\
	0\arrow[r]&H_q(F)\arrow[r]&H_{q}(E)\arrow[r]&H_{q-n}(F)\arrow[r]&0
\end{tikzcd}\]\fi
This is the other "half" of the Wang sequence. It only remains to put both halves together and conclude that
\[\begin{tikzcd}[column sep=small]
	\cdots\arrow[r]&H_{q}(F)\arrow[r,"i^*"]&H_{q}(E)\arrow[r]&E^{\infty}_{n,q-n}\arrow[r,"d^n_{n,q-n}"]&H_{q-n}(F)\arrow[r]&H_{q-1}(F)\arrow[r]&E^\infty_{0,q-1}\arrow[r]&\cdots
\end{tikzcd}\]
But we may remove the $E^\infty$ terms to get
\[\begin{tikzcd}[column sep=small]
	\cdots\arrow[r]&H_{q}(F)\arrow[r,"i^*"]&H_{q}(E)\arrow[r]&H_{q-n}(F)\arrow[r,"d^n_{n,q-n}"]&H_{q-1}(F)\arrow[r]&H_{q-1}(E)\arrow[r]&H_{q-n-1}(F)\arrow[r]&\cdots
\end{tikzcd}\]
\subsection{Extra exercise on spectral sequences}\label{ssec:Extra exercise on spectral sequences}
\begin{exercise}[June 20]
	Let $F\hookrightarrow E\to B$ be a Serre fibration, $\mathbb{F}$ a field and suppose the $\pi_{1}(B)\curvearrowright  H_{*}(F,\mathbb{F})$ is trivial. Suppose $\chi(B)$ and $\chi(F)$ exist. Show that $\chi(E)=\chi(F)\cdot \chi(B)$.
\end{exercise}
\begin{proof}[Solution]
	Recall that for a topological space $X$,
	\[\chi(X)=\sum_{i}(-1)^i\dim H_{i}(X,\mathbb{F}).\]
	Also recall that when the coefficients in a spectral sequence are a field, we have the following nice expression for the associated graded:
	\[H_{n}(X)\cong \bigoplus_{p}E^\infty_{p,n-p}=\bigoplus_{p} F_{p}H_{n}/F_{p-1}H_{n} \]
	Then we have that 
	\begin{align*}
		\chi(E)&=\sum_{i}(-1)^i\dim H_{i}(E)\\
		       &=(-1)^i\sum_{i}\dim \left( \bigoplus_{p}F_{p}/F_{p-1}   \right)\\
		       &=(-1)^i\sum_{i}\dim E^\infty_{p,n-p}
	\end{align*}
	On the other hand, we have that
	\[\chi(F)\cdot \chi(B)=\sum_{i}(-1)^i\dim H_{i}(F)\cdot \sum_{j}(-1)^{j}\dim H_{j}(B)={\color{magenta}\sum_{i,j}(-1)^{i+j}E^r_{j,i}}.\]
	If we understand the last equation, our exercise is solved once we show that
	\begin{claim}
		The expression $\sum_{i,j}(-1)^{i+j}E^r_{j,i}$	does note depend on $r$.
	\end{claim}
	

	Here's a few facts that will turn out helpful
	\begin{itemize}
		\item $H_{i}(X,\mathbb{F}^k)\cong \bigoplus_{k} H_{i}(X,\mathbb{}) $
		\item $H_{i}(X,V)=V\otimes H_{i}(X,\mathbb{F})$ for a vector space $V$ over $\mathbb{F}$.
		\item $H_{p}(B,H_{q}(F,\mathbb{F}))\cong H_{q}(F,\mathbb{F})\otimes H_{p}(B,\mathbb{F})$.
		\item For a chain complex 
			\[\begin{tikzcd}[column sep=small]
				\cdots \arrow[r]&C_{k}\arrow[r]&C_{k-1}\arrow[r]&\cdots
			\end{tikzcd}\]
			we have
			\[\sum_{i}(-1)^i\dim C_i=\sum_{i}(-1)^i\dim H_{i}(C_{\bullet})\]
	\end{itemize}
	
\end{proof}

\section{Exam questions}\label{sec:Exam questions}
\subsection{Computation of $\pi_{4}(S^{3})$}
\subsubsection{Solution from  \href{https://math.stackexchange.com/questions/1102897/computing-pi-4s3-using-serre-spectral-sequence}{StackExchange}}


\subsubsection{Solution from \href{https://people.math.wisc.edu/~lmaxim/754notes.pdf}{754notes.pdf}}
When we apply the Postnikov tower to $S^{3}$, we obtain from the third and fourth levels a fibration
\[K(\pi_{4}(S^{3}),4)\hookrightarrow Y_{4}\to Y_3=K(\mathbb{Z},3).\]
Now we do the homology spectral sequence of this fibration. We get the following second page
	\begin{align*}
		\begin{array}{c|c c c c}
			4&\pi_{4}&&&\pi_{4}\\
			3&&&&\\
			2&&&&\\
			1&&&&\\
			0&\pi_4&&&\pi_4\\
			\hline
			E^{2}&0&1&2&3
		\end{array}
	\end{align*}
	where $\pi_4=\pi_{4}(S^{3})$. But we don't know what lies further up nor right.

	We should look at the 5th page, since the differential
	\[d^{5}_{5,0}:E^{5}_{5,0}\to E^{5}_{0,4}\]
	has chances of being non-trivial.
	\begin{align*}
		\begin{array}{c|c c c c c c}
			4&\pi_{4}&&&\pi_{4}&&\\
			3&&&&&&\\
			2&&&&&&\\
			1&&&&&&\\
			0&\pi_4&&&\pi_4&&?\\
			\hline
			&0&1&2&3&4&5
		\end{array}
	\end{align*}

	Recall that a CW decomposition of $S^{3}$ is given by a 3-cell and a 0-cell. Since $Y_4$ is obtained from $S^{3} $ by attaching cells of dimension $n+2$, we get that
\[H_{4}(Y_{4})=H_{5}(Y_5)=0.\]
We know there is a filtration on homology
\[0=F_{-1}H_{n}\subset F_{0}H_{n}\subset \ldots\subset F_{n}H_n=H_n(Y_4)\]
such that
\[E^{\infty}_{p,q}=F^{p} H^{p+q} /F^{p-1} H^{p+q}.\]
But the 4th and 5th homology groups are zero, so the filtration is zero and so are the groups $E^{\infty}_{p,q}$ with $p+q=4,5$. And of course the groups $E^{6}_{5,0}$ and $E^{6}_{0,4}$ are (always) already $E^{\infty}_{5,0}$ and $E^{\infty}_{0,4}$, meaning
\[0=E^{6}_{5,0}=\dfrac{\ker d^5}{\img d^{5}}{\color{red}\implies }\pi_{4}(S^{3})\cong H_{5}(K(\mathbb{Z},3)).\]

…so finally we are convinced that we are looking for $H^{5}(K(\mathbb{Z},3),\mathbb{Z})$.

To find such a group let's compute the cohomology spectral sequence of the fibration
\[\Omega K(\mathbb{Z},3)\hookrightarrow PK(\mathbb{Z},3)\to K(\mathbb{Z},3).\]
We immediately notice $\pi_{i}(\Omega K(\mathbb{Z},3))\cong \pi_{i+1}(K(\mathbb{Z},3))$ by the long exact sequence in homotopy. So, another name for $\Omega K(\mathbb{Z},3)$ is $K(\mathbb{Z},2)$ which in turn is just $\mathbb{C}P^{\infty}$. To find out how the second page looks like we use the fact that homology groups of $\mathbb{C}P^{\infty}$ are finitely generated, {\color{magenta}so that by the universal coefficients theorem we have
\[E^{p,q}_{2}=H^{p}(K(\mathbb{Z},3);H^{q}(\mathbb{C}P^{\infty}))\cong H^{p}(K(\mathbb{Z},3))\otimes H^{q}(\mathbb{C}P^{\infty}).\]}
This gives us
\begin{align*}
\begin{array}{c|c c c c c c c}
	4&a^{2}\mathbb{Z}&&&s\mathbb{Z}\otimes a^{2} \mathbb{Z}&&&\\
	3&&&&&&&\\
	2&a\mathbb{Z}&&&s\mathbb{Z}\otimes a\mathbb{Z}&&&\\
	1&&&&&&&\\
	0&\mathbb{Z}&&&s\mathbb{Z}\otimes \mathbb{Z}&&?&?\\
	\hline
	 &0&1&2&3&4&5&6
\end{array}
\end{align*}
where we have chosen generators for each cohomology group in $H^{\bullet}(K(\mathbb{Z},3))\otimes H^{\bullet}(\mathbb{C}P^{\infty})$ according to its product structure.

Our interest in groups marked with a question mark is justified as follows. Again by the \href{https://en.wikipedia.org/wiki/Universal_coefficient_theorem#Universal_coefficient_theorem_for_cohomology}{universal coefficient theorem for cohomology} we have a short exact sequence
\[\begin{tikzcd}
	0\arrow[r]&\opertatorame{Ext}(H_{i-1}(X;R),G)\arrow[r]&H^{i}(X;G)\arrow[r,"h"]&\opertatorame{Hom}(H_{i}(X;R),G)\arrow[r]&0
\end{tikzcd}\]
which splits, giving for $i=5$
\[H^{5}=\opertatorame{Hom}(H_{5},\mathbb{Z})\oplus \opertatorame{Ext}(H_{4},\mathbb{Z}),\]
where the homology and cohomolgy groups correspond to $K(\mathbb{Z},3)$.

Since cohomology groups are finitely generated, {\color{magenta}we may write
\[H_{5}=\text{Free part of } H^{5}\oplus \text{Torsion part of } H^{6}.\]}
Now let's try to compute those groups. First notice that since $E_{2}^{p,q}=0$ for $q$ odd, we have that $d_{2}=0$, so that $E_{2}= E_{3}$. The same happens for all even pages, i.e. $E_{2k}=E_{2k+1}$.

To find $H^{6}$ notice that
\[d_{3}^{3,2}:E^{3,2}_{3}\to E^{6,0}_{3}=H^{6}(K(\mathbb{Z},3))\otimes \mathbb{Z}\] 
maps
\[s\otimes a\longmapsto ds\otimes a-s\otimes  da=0\otimes  a\pm s^{2}\]
which is clear by the other differentials in this page. {\color{magenta}And because the generator squared is minus the generator, the group $H^{5}$ must be $\mathbb{Z}/2$. And $H^{6}$ is zero.}

\subsection{Computation of $\pi_{5}(S^{3})$}
Actually, the following argument reproves the previous exercise.

Consider the Whitehead tower of $S^{3}$:
\[\begin{tikzcd}
	K(\mathbb{Z},3)\arrow[r]&X_4\arrow[d]\\
	K(\mathbb{Z},2)\arrow[r]&X_3\arrow[d]\\
				&S^{3}
\end{tikzcd}\]
and quickly notice that
\[H_{5}(X_4)=\pi_{5}(X_4)=\pi_{5}(S^{3})\]
and that
\[H_{4}(X_3)=\pi_{4}(X_3)=\pi_{4}(S^{3})\]
because of the connectedness of $X_{3}$ and $X_{4}$.

Now let's do cohomology spectral sequence for the second fibration. Recall that $K(\mathbb{Z},2)\cong \mathbb{C}P^{\infty}$. And because all groups are free, we get that
\[ E^{p,q}_{2}=H^{p}(S^{3},\mathbb{Z})\otimes H^{q}(\mathbb{C}P^{\infty},\mathbb{Z})\]
so that the second page is
\begin{align*}
\begin{array}{c|c c c c c}
	4&\mathbb{Z}\otimes \mathbb{Z}&&&\mathbb{Z}\otimes \mathbb{Z}&\\
	3&&&&&\\
	2&\mathbb{Z}\otimes \mathbb{Z}&&&\mathbb{Z}\otimes \mathbb{Z}&\\
	1&&&&&\\
	0&\mathbb{Z}\otimes \mathbb{Z}&&&\mathbb{Z}\otimes \mathbb{Z}&\\
	\hline
	 &0&1&2&3&4&
\end{array}
\end{align*}
{\color{magenta}(are these tensor products really correct? Because in the pdf we have only $\mathbb{Z}$…)} So we see that $E^{2}=E^{3}$ and $E^{4}=E^{\infty} $.  

We also know by Hurewicz {\color{magenta}(but of course Hurewicz is on \textit{homology}, telling us that second and third homology groups vanish, so really we should be using universal coefficient theorem to get vanishing cohomology from that… so perhaps we use that \textit{first} and \textit{second} homology vanish to get that second and third cohomology vanish…)} that
\[H^{2}(X_{3})=H^{3}(X_{3})=0\]
which means that the groups in the $E_{\infty}=E^{4}$ page are zero (because they are subsequent quotients of a filtration on homology), and this in turn means that the differentials on $E^{3}$ are isomorphisms.

Now let's see what we get from those isomorphisms. We use the product structure on $H^{\bullet}(\mathbb{C}P^{\infty})\cong \mathbb{Z}[x]$ where $x^{2}=1$. We re-write the second page using the generators of each group like this:
\begin{align*}
\begin{array}{c|c c c c c}
	4&x^{2}&&&s^{3}x&\\
	3&&&&&\\
	2&x&&&s^{2}&\\
	1&&&&&\\
	0&1&&&s&\\
	\hline
	 &0&1&2&3&4&
\end{array}
\end{align*}
And then we know that we may choose the generators so that $d_{3}$ maps $x\mapsto s$ (because we have seen that $d_{3}$ is an isomorphism). Then we get by Leibliz property of the product that
\[d_{3}x^{n} =nx^{n-1} dx=nx^{n-1} u\]
Where $x^{n} $ is the generator of $E^{0,2n}_{3}$ on the left column and $x^{n-1}u$ is the generator of the corresponding group on the right column. The last equation says that differential $d^{0,2n}_{3}$ maps a generator to $n$ times the generator. That is, it is multiplication by $n$.

Then it very easily follows that the groups on the fourth page, being quotients by the kernel of each differential, are $\mathbb{Z}/k$ coming from the $2k$-th row. And because everything else in the filtration is zero, these groups are the cohomology groups of $X_{3}$, and finally by universal coefficients we know the homology groups.

[Put table here, but it says that for big $k$, the $2k+1$-th cohomology group is  $\mathbb{Z}/k$ and the $2k$ homology is  $\mathbb{Z}/k$.
\subsection{$\pi_{i}(S^{2k+1})$ is finite for $i>2k+1$}
This proof is an induction on $k$. The case $k=0$ is easy since $\pi_{1}(S^{1})$ is trivial (since its universal cover is contractible). For other $k$ we will use the fact proved in lectures that the groups $\pi_{i}(S^{2k+1})$ are finitely generated (and abelian) for all $i>1$. This means that they are finite if they are torsion groups. This exercise is proved once we show that
\begin{equation}\label{eq:1}
	\pi_{i}(S^{2k-1})\cong \pi_{i+2}(S^{2k+1})\quad \mod\text{torsion}.
\end{equation}
This will work since it shows that, for instance, $\pi_{i+2}(S^{3})$ are congruent {\color{magenta}modulo torsion} to $\pi_{i}(S^{1})$ for all $i$, which  {\color{magenta}makes them torsion groups?}.

The trick is proving that
\begin{equation}\label{eq:2}
	\pi_{2k-1}(\Omega^{2} S^{2k+1})\cong \pi_{2k+1}(S^{2k+1})\cong \mathbb{Z}.
\end{equation}
The hardest part is showing that a generator of $\pi_{2k-1}(\Omega^{2} S^{2k+1})$, which is a map $\beta:S^{2k-1}\to \Omega^{2} S^{2k+1}$, induces an isomorphism mod torsion on $H_{*}$, {\color{magenta}which is just another way of saying that it is an isomorphism on $H_{*}(-;\mathbb{Q})$}.

When that is proved, the homology long exact sequence of the pair $\Omega^{2}S^{2k+1},S^{2k-1}$ says that
\[H_\bullet(\Omega^{2} S^{2k+1},S^{2k-1})=0\mod \text{torsion} \]
and by relative Hurewicz mod torsion theorem we have that
\[\pi_{i}(\Omega^{2} S^{2k+},S^{2k-1})=0\mod\text{torsion}\; \forall i>0\]
giving the desired isomorphism by the long exact sequence in homotopy.

\subsection{$\pi_{i}(S^{2k})$ is finite for $i>2k$ and $i\neq 4k-1$}
Our objective will be to obtain a contradiction to the hairy-ball theorem by constructing a nowhere-vanishing vector field on $S^{2k}$.

We shall define $E$ to be the tangent bundle of $S^{2k}$  \textit{without} all zero vectors, yielding a fibration
\[S^{2k-1}\hookrightarrow E\to S^{2k}\]
since the fiber is $\mathbb{R}^{2k}\backslash \{0\}\simeq S^{2k-1}$. Then the Serre homology spectral sequence reads
\[E^{2}_{p,q}=H_{p}(S^{2k},H_{q}(S^{2k-1}){\color{magenta}=}H_{p}(S^{2k})\otimes H_{q}(S^{2k-1})\]
so that there's only four non-trivial entries on the second page:
\begin{align*}
\begin{array}{c|c c c c c c}
	2k-1&\mathbb{Z}&&&&\mathbb{Z}&\\
	\vdots &&&&&&\\
	3&&&&&&\\
	2&&&&&&\\
	1&&&&&&\\
	0&\mathbb{Z}&&&&\mathbb{Z}&\\
	\hline
	&0&1&2&\cdots &2k&
\end{array}
\end{align*}
so that only the differential $d^{2k}_{2k,0}$ on the $2k$-th page can be non-trivial. And we have that $E^{2}= \ldots=E^{2k}$ and $E^{2k+1}=\ldots=E^{\infty}$.

Our objective shall be to show that
\[H_{i}(E)=\begin{cases}
	\text{finite} \qquad &i\neq 0,\; 4k-1\\
	\mathbb{Z} \oplus \text{finite} \qquad &i=4k-1.
\end{cases}\]
{\color{magenta}This will hold if $d^{2k}_{2k,0}\neq 0$.}

In search for a contradiction suppose this differential vanishes. {\color{magenta}Notice} the following diagram commutes:
\[\begin{tikzcd}
	\pi_{2k}(S^{2k})\arrow[r,"\partial"]\arrow[d,swap,"h"]\arrow[d,"\cong "]&\pi_{2k-1}(S^{2k-1})\arrow[d,"h"]\arrow[d,"\cong ",swap]\\
	H_{2k}(S^{2k})\arrow[r,"d^{2k}",swap]&H_{2k-1}(S^{2k-1})
\end{tikzcd}\]
where $\partial$ is the connection homomorphism in the homotopy long exact sequence. This means that $d^{2k}=0$ iff $\partial=0$.

Given our hypothesis that $d^{2k}=0$, we get a surjection
\[\begin{tikzcd}
	\pi_{2k}(E)\arrow[r,"pi_{*}"]&\pi_{2k}(S^{2k})\arrow[r,"\partial"]&0
\end{tikzcd}\]
So, there is a homotopy class $[\phi]\in \pi_{2k}(E)$ mapping to $[\operatorname{id}]$ under $\pi^{*}$. But then $\phi$ is a map $S^{2k}\to E$ and $\pi^{*}[\phi]$ is obtained by precomposing with $\phi$, so we have a diagram
\[\begin{tikzcd}
	&E\arrow[d,"\pi"]\\
	S^{2k}\arrow[ur,"\phi"]\arrow[r,equals,"\operatorname{id}"]&S^{2k}
\end{tikzcd}\]
that commutes {\color{magenta}up to homotopy}. This means that by the {\color{magenta}homotopy lifting property} there is a map $\psi:S^{2k}\to E$ lifting the identity, that is, a nowhere vanishing vector field on $S^{2k}$, which is a contradiction.

\clearpage
\addcontentsline{toc}{section}{References}
\printbibliography
\clearpage
\end{document}
